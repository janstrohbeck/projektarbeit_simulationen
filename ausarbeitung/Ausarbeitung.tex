\documentclass[oneside]{elaboration}
\newmintedfile{qml}{frame=lines,linenos,mathescape,framesep=2mm,breaklines,fontsize=\footnotesize}
\newmint{qml}{mathescape}
\newmintinline{qml}{mathescape}
\newminted{qml}{mathescape,breaklines}

% Pfad für Eingabedateien
\makeatletter
\def\input@path{{chapters/}}
\makeatother
\graphicspath{{./images/},{../uml/build/}}

\usepackage{relsize}
%c from texinfo.tex
\def\ifmonospace{\ifdim\fontdimen3\font=0pt }

% Deckblatt

\begin{document}

\author{Jan Strohbeck}
\title{\textbf{Simulation von Echtzeit-Prozessen in Ada und C/POSIX}}
\doctype{Projektarbeit}
\examinerA{Prof. Dr. Roland Dietrich}
\date{27.\ März 2017}

\maketitle
\newpage

% Inhaltsverzeichnis

\tableofcontents
\newpage

\pagenumbering{Roman}
\setcounter{page}{1}

% Abbildungsverzeichnis

\listoffigures
\addcontentsline{toc}{chapter}{Abbildungsverzeichnis}
\newpage

\pagenumbering{arabic}
\setcounter{page}{1}

% Use page style with header for chapters starting from here
\renewcommand*{\chapterpagestyle}{scrheadings}

% Einleitung
\chapter{Einleitung}
\label{chp:einleitung}

In der vorliegenden Projektarbeit wurden zwei Simulationsprogramme erstellt,
die reale Umgebungen simulieren sollen. Diese können später von Studenten dazu
verwendet werden, Problemstellungen innerhalb dieser Umgebungen zu lösen. Die
Umgebungen sind für die Vorlesung \enquote{Echtzeitsysteme} gedacht, in welcher
die Studenten unter anderem lernen, nebenläufige Programme sowie die
Kommunikation unter mehreren Threads zu verstehen und selbst zu entwickeln.
Dabei kommen die Programmiersprachen Ada sowie C unter Verwendung der
POSIX-Schnittstelle zum Einsatz. Die Simulationsumgebungen wurden daher jeweils in
Ada und C umgesetzt, sodass die Studenten die Programmierung mit beiden
Programmiersprachen bzw.\ Schnittstellen üben und vergleichen können.

Im folgenden Kapitel werden zunächst die Grundlagen über die Programmierung von
nebenläufigen Programmen in Ada sowie C/POSIX dargestellt. Danach wird in
Kapitel \ref{chp:uebersicht_ueber_das_projekt} eine Übersicht über die Inhalte
der Projektarbeit sowie die simulierten Umgebungen gegeben. In Kapitel
\ref{chp:umsetzung} wird die Umsetzung des Projekts besprochen. Abschließend
werden in den Kapiteln \ref{chp:diskussion} und \ref{chp:fazit} die Ergebnisse
der Arbeit diskutiert sowie ein abschließendes Fazit gebildet.

% Grundlagen
\chapter{Grundlagen}
\label{chp:grundlagen}

In diesem Kapitel werden die Möglichkeiten zur Programmierung von nebenläufigen
Programmen in Ada sowie C/POSIX aufgezeigt.

% - C/POSIX
\section{C/POSIX}
\label{sec:c/posix}

In der Programmiersprache C existieren selbst keine eingebauten
Funktionalitäten, um parallele Abläufe umzusetzen. Programme werden prinzipiell
in nur einem Thread ausgeführt. Um nebenläufige Threads verwenden zu können,
muss auf Bibliotheken zurückgegriffen werden. Die gängigsten Betriebssysteme
unterstützten mehrere Threads pro Prozess und bieten Bibliotheken, um diese in C
nutzen zu können. POSIX, welches eine standardisierte Schnittstelle für
Betriebssystem-Aufrufe darstellt, definiert Betriebssystem-Aufrufe, um innerhalb
eines Prozesses mehrere Threads starten zu können. Um diese Aufrufe in C tätigen
zu können, wird meist die \texttt{pthread}-Bibliothek verwendet, welche auf
den meisten gängigen Betriebssystemen verfügbar ist. Damit können dann prinzipiell
plattformunabhängig nebenläufige Programme entwickelt werden.

POSIX und die \texttt{pthread}-Bibliothek definieren auch Konstrukte für die
Synchronisierung mehrerer Threads, z.B.\ über Semaphoren, Mutexe,
Bedingungsvariablen und Message Queues. Dies sind sehr systemnahe Konstrukte, es
wird dabei viel Verantwortung dem Programmierer übertragen. Falsche Verwendung
dieser Funktionen kann zu selten auftretenden und dadurch schwer auffindbaren
Fehlern führen. Dennoch erlaubt die Systemnähe viel Kontrolle über den
Programmablauf und kann effiziente Lösungen hervorbringen.

% - Ada
\section{Ada}
\label{sec:ada}

Die Programmiersprache Ada unterstützt nebenläufige Threads als sog.\
\enquote{Tasks} als Sprachkonstrukt. Synchronisierung von Tasks und
Nachrichtenaustausch zwischen diesen ist ebenso mit Sprachkonstrukten möglich.
Es sind somit keine zusätzlichen Bibliotheken notwendig, es werden vom
Ada-Compiler automatisch die vom Betriebssystem zur Verfügung gestellten
Möglichkeiten zur Implementierung von Threads verwendet, oder über eine
Laufzeitumgebung bereitgestellt (falls das Betriebssystem keine Threads
unterstützt oder kein Betriebssystem vorhanden ist). Dadurch, dass Tasks,
Synchronisation und Nachrichtenaustausch in die Sprache integriert sind,
übernimmt die Programmiersprache die Verantwortung zur korrekten Umsetzung der
so definierten Konstrukte. Damit können viele Arten von Fehlern direkt vermieden
werden. Andererseits besitzt der Programmierer somit nicht mehr so viel
Kontrolle über die genauen Abläufe in Programm, da diese abstrahiert wurden.

% Übersicht über das Projekt
\chapter{Übersicht über das Projekt}
\label{chp:uebersicht_ueber_das_projekt}

Hier soll ein kurzer Überblick über die einzelnen Teilsimulationen gegeben
werden sowie über die Anforderungen an die Umsetzung dieser Simulationen.

% - Druck- und Temperatursteuerung
\section{Druck- und Temperatursteuerung}
\label{sec:druck-_und_temperatursteuerung}

Hier wird ein System simuliert, in welchem sich Druck und Temperatur dynamisch
ändern. Darin soll es ein eingebettetes System geben, welches diese beiden
Größen mithilfe von Sensoren messen kann und durch Aktoren (Heizung, Ventil)
beeinflussen kann. Des Weiteren kann das eingebettete System die aktuellen Werte
des Messgrößen auf ein Display schreiben. Dies ist in dem Diagramm in
Abb.~\ref{fig:simple_embedded_system_overview} dargestellt.

\begin{figure}[hbt]
\centering
\includegraphics[width=0.9\textwidth]{simple_embedded_system_overview}
\caption{Übersicht über die Druck- und Temperatursteuerung}
\label{fig:simple_embedded_system_overview}
\end{figure}

Die Implementierung des eingebetteten Systems soll als Übung für die Studenten
geschehen. Damit dies möglichst einfach ist, soll im Rahmen dieser Projektarbeit
ein Grundgerüst geschaffen werden, welches Werte für Druck und Temperatur
generiert, sowie die Methoden zum Abfragen dieser Werte sowie zum Ansteuern der
simulierten Aktoren bereitstellt. Die Ansteuerung der Aktoren soll eine
nachvollziehbare und einigermaßen realistischen Wirkung auf die Veränderung der
Werte von Druck und Temperatur haben, sodass eine Regelung auf einen Sollwert
möglich ist. Die Regelung selbst soll ebenfalls bereitgestellt werden, da die
Implementierung von Regelungen nicht Teil der Übungen sein soll.

% - Parkplatz-Steuerung
\section{Parkplatz-Steuerung}
\label{sec:parkplatz-steuerung}

In dieser Übung soll ein einfacher Parkplatz simuliert werden, den Autos
über zwei Schranken befahren oder verlassen. Außerdem soll eine Signallampe
existieren, welche anzeigt, ob der Parkplatz voll ist oder nicht.

Auch hier soll es ein eingebettetes System geben, welches die Ansteuerung der
Schranken und des Signals übernimmt. Dafür kann es auf mehrere Sensoren
zurückgreifen, welche ihm mitteilen, ob gerade Autos vor einer Schranke warten
und ob gerade Autos unter einer Schranke hindurchfahren, wenn diese geöffnet
ist. Abb.~\ref{fig:parking_lot_overview} stellt dieses System in einem Diagramm
dar.

\begin{figure}[hbt]
\centering
\includegraphics[width=\textwidth]{parking_lot_overview}
\caption{Übersicht über die Parkplatz-Simulation}
\label{fig:parking_lot_overview}
\end{figure}



% - Dokumentation
\section{Dokumentation}
\label{sec:dokumentation}

% - Sonstiges
\section{Sonstiges}
\label{sec:sonstiges}

Anpassbarkeit der Log-Meldungen

% Umsetzung
\chapter{Umsetzung}
\label{chp:umsetzung}

% - Ada
\section{Ada}
\label{sec:ada}

% - C/POSIX
\section{C/POSIX}
\label{sec:c/posix}

% - Dokumentation
\section{Dokumentation}
\label{sec:dokumentation}

Sphinx

Python 2

sphinxcontrib-adadomain

Eigene adadomain.py

% Diskussion
\chapter{Diskussion}
\label{chp:diskussion}

% Fazit
\chapter{Fazit}
\label{chp:fazit}

% Bibliografie

\clearpage
%\pagestyle{plain}
%\renewcommand*{\chapterpagestyle}{plain}

%\nocite{*}
\addcontentsline{toc}{chapter}{Literatur}
\printbibliography

\appendix{}
% Anhang
\chapter{Anhang}
\label{chp:anhang}

% - Projekt-Dokumentation
\section{Projekt-Dokumentation}
\label{sec:projekt-dokumentation}

\end{document}
