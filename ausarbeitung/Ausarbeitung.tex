\documentclass[oneside]{elaboration}
\usepackage[strings]{underscore}
\newminted{bash}{autogobble,xleftmargin=30pt}

% Pfad für Eingabedateien
\makeatletter
\def\input@path{{chapters/}}
\makeatother
\graphicspath{{./images/},{../uml/build/}}

\usepackage{relsize}
%c from texinfo.tex
\def\ifmonospace{\ifdim\fontdimen3\font=0pt }

% Deckblatt

\begin{document}

\author{Jan Strohbeck}
\title{\textbf{Simulation von Echtzeit-Prozessen in Ada und C/POSIX}}
\doctype{Projektarbeit}
\examinerA{Prof. Dr. Roland Dietrich}
\date{27.\ März 2017}

\maketitle
\newpage

% Inhaltsverzeichnis

\tableofcontents
\newpage

\pagenumbering{Roman}
\setcounter{page}{1}

% Abbildungsverzeichnis

\listoffigures
\addcontentsline{toc}{chapter}{Abbildungsverzeichnis}
\newpage

\pagenumbering{arabic}
\setcounter{page}{1}

% Use page style with header for chapters starting from here
\renewcommand*{\chapterpagestyle}{scrheadings}

% Einleitung
\chapter{Einleitung}
\label{chp:einleitung}

In der vorliegenden Projektarbeit wurden zwei Simulationsprogramme erstellt,
die reale Umgebungen simulieren sollen. Diese können später von Studenten dazu
verwendet werden, Problemstellungen innerhalb dieser Umgebungen zu lösen. Die
Umgebungen sind für die Vorlesung \enquote{Echtzeitsysteme} gedacht, in welcher
die Studenten unter anderem lernen, nebenläufige Programme sowie die
Kommunikation unter mehreren Threads zu verstehen und selbst zu entwickeln.
Dabei kommen die Programmiersprachen Ada sowie C unter Verwendung der
POSIX-Schnittstelle zum Einsatz. Die Simulationsumgebungen wurden daher jeweils in
Ada und C umgesetzt, sodass die Studenten die Programmierung mit beiden
Programmiersprachen bzw.\ Schnittstellen üben und vergleichen können.

Im folgenden Kapitel werden zunächst die Grundlagen über die Programmierung von
nebenläufigen Programmen in Ada sowie C/POSIX dargestellt. Danach wird in
Kapitel \ref{chp:uebersicht_ueber_das_projekt} eine Übersicht über die Inhalte
der Projektarbeit sowie die simulierten Umgebungen gegeben. In Kapitel
\ref{chp:umsetzung} wird die Umsetzung des Projekts besprochen. Abschließend
werden in den Kapiteln \ref{chp:diskussion} und \ref{chp:fazit} die Ergebnisse
der Arbeit diskutiert sowie ein abschließendes Fazit gebildet.

% Grundlagen
\chapter{Grundlagen}
\label{chp:grundlagen}

In diesem Kapitel werden die Möglichkeiten zur Programmierung von nebenläufigen
Programmen in Ada sowie C/POSIX aufgezeigt.

% - C/POSIX
\section{C/POSIX}
\label{sec:c/posix}

In der Programmiersprache C existieren selbst keine eingebauten
Funktionalitäten, um parallele Abläufe umzusetzen. Programme werden prinzipiell
in nur einem Thread ausgeführt. Um nebenläufige Threads verwenden zu können,
muss auf Bibliotheken zurückgegriffen werden. Die gängigsten Betriebssysteme
unterstützten mehrere Threads pro Prozess und bieten Bibliotheken, um diese in C
nutzen zu können. POSIX, welches eine standardisierte Schnittstelle für
Betriebssystem-Aufrufe darstellt, definiert Betriebssystem-Aufrufe, um innerhalb
eines Prozesses mehrere Threads starten zu können. Um diese Aufrufe in C tätigen
zu können, wird meist die \texttt{pthread}-Bibliothek verwendet, welche auf
den meisten gängigen Betriebssystemen verfügbar ist. Damit können dann prinzipiell
plattformunabhängig nebenläufige Programme entwickelt werden.

POSIX und die \texttt{pthread}-Bibliothek definieren auch Konstrukte für die
Synchronisierung mehrerer Threads, z.B.\ über Semaphoren, Mutexe,
Bedingungsvariablen und Message Queues. Dies sind sehr systemnahe Konstrukte, es
wird dabei viel Verantwortung dem Programmierer übertragen. Falsche Verwendung
dieser Funktionen kann zu selten auftretenden und dadurch schwer auffindbaren
Fehlern führen. Dennoch erlaubt die Systemnähe viel Kontrolle über den
Programmablauf und kann effiziente Lösungen hervorbringen.

% - Ada
\section{Ada}
\label{sec:ada}

Die Programmiersprache Ada unterstützt nebenläufige Threads als sog.\
\enquote{Tasks} als Sprachkonstrukt. Synchronisierung von Tasks und
Nachrichtenaustausch zwischen diesen ist ebenso mit Sprachkonstrukten möglich.
Es sind somit keine zusätzlichen Bibliotheken notwendig, es werden vom
Ada-Compiler automatisch die vom Betriebssystem zur Verfügung gestellten
Möglichkeiten zur Implementierung von Threads verwendet, oder über eine
Laufzeitumgebung bereitgestellt (falls das Betriebssystem keine Threads
unterstützt oder kein Betriebssystem vorhanden ist). Dadurch, dass Tasks,
Synchronisation und Nachrichtenaustausch in die Sprache integriert sind,
übernimmt die Programmiersprache die Verantwortung zur korrekten Umsetzung der
so definierten Konstrukte. Damit können viele Arten von Fehlern direkt vermieden
werden. Andererseits besitzt der Programmierer somit nicht mehr so viel
Kontrolle über die genauen Abläufe in Programm, da diese abstrahiert wurden.

% Übersicht über das Projekt
\chapter{Übersicht über das Projekt}
\label{chp:uebersicht_ueber_das_projekt}

Hier soll ein kurzer Überblick über die einzelnen Teilsimulationen gegeben
werden sowie über die Anforderungen an die Umsetzung dieser Simulationen.

% - Druck- und Temperatursteuerung
\section{Druck- und Temperatursteuerung}
\label{sec:druck-_und_temperatursteuerung}

Hier wird ein System simuliert, in welchem sich Druck und Temperatur dynamisch
ändern. Darin soll es ein eingebettetes System geben, welches diese beiden
Größen mithilfe von Sensoren messen kann und durch Aktoren (Heizung, Ventil)
beeinflussen kann. Des Weiteren kann das eingebettete System die aktuellen Werte
des Messgrößen auf ein Display schreiben. Dies ist in dem Diagramm in
Abb.~\ref{fig:simple_embedded_system_overview} dargestellt.

\begin{figure}[hbt]
\centering
\includegraphics[width=0.9\textwidth]{simple_embedded_system_overview}
\caption{Übersicht über die Druck- und Temperatursteuerung}
\label{fig:simple_embedded_system_overview}
\end{figure}

Die Implementierung des eingebetteten Systems soll als Übung für die Studenten
geschehen. Damit dies möglichst einfach ist, soll im Rahmen dieser Projektarbeit
ein Grundgerüst geschaffen werden, welches Werte für Druck und Temperatur
generiert, sowie die Methoden zum Abfragen dieser Werte sowie zum Ansteuern der
simulierten Aktoren bereitstellt. Die Ansteuerung der Aktoren soll eine
nachvollziehbare und einigermaßen realistischen Wirkung auf die Veränderung der
Werte von Druck und Temperatur haben, sodass eine Regelung auf einen Sollwert
möglich ist. Die Regelung selbst soll ebenfalls bereitgestellt werden, da die
Implementierung von Regelungen nicht Teil der Übungen sein soll.

% - Parkplatz-Steuerung
\section{Parkplatz-Steuerung}
\label{sec:parkplatz-steuerung}

In dieser Übung soll ein einfacher Parkplatz simuliert werden, den Autos
über zwei Schranken befahren oder verlassen. Außerdem soll eine Signallampe
existieren, welche anzeigt, ob der Parkplatz voll ist oder nicht. Die Autos
können sich vor den Schranken in eine Warteschlange einreihen, sie werden dann
in der Reihenfolge der Ankunft bedient. Falls sie nicht innerhalb von 30
Sekunden eingelassen werden, verlassen die Autos die Warteschlange wieder (nur
bei der Schranke am Eingang).

Auch hier soll es ein eingebettetes System geben, welches die Ansteuerung der
Schranken und des Signals übernimmt. Dafür kann es auf mehrere Sensoren
zurückgreifen, welche ihm mitteilen, ob gerade Autos vor einer Schranke warten
und ob gerade Autos unter einer Schranke hindurchfahren, wenn diese geöffnet
ist. Abb.~\ref{fig:parking_lot_overview} stellt dieses System in einem Diagramm
dar.

\begin{figure}[hbt]
\centering
\includegraphics[width=\textwidth]{parking_lot_overview}
\caption{Übersicht über die Parkplatz-Simulation}
\label{fig:parking_lot_overview}
\end{figure}

Das eingebettete System soll von den Studenten unter Verwendung eines
bereitgestellten Grundgerüsts als Übung durchgeführt werden. Dieses Grundgerüst
soll mehrere Autos simulieren, welche sich in Warteschlangen vor den Schranken
einreihen. Über bereitgestellte Methoden soll es für die Studenten möglich sein,
die Sensoren abzufragen, sowie die Schranken und das Signal anzusteuern. Damit
soll dann eine realistische Steuerung des Parkplatzes möglich sein.

% - Dokumentation
\section{Dokumentation}
\label{sec:dokumentation}

Damit die Verwendung für die Studenten möglichst einfach ist, soll eine
umfangreiche Dokumentation erstellt werden, welche die notwendigen Schritte zur
Verwendung der Simulationen aufführt sowie eine Übersicht über die
bereitgestellten Methoden mitsamt Beschreibungen enthält. Für interessierte
Studenten sollen auch die Details der Implementierung in die Dokumentation
aufgenommen werden.

% - Sonstiges
\section{Sonstiges}
\label{sec:sonstiges}

Beide Simulationen sollen außerdem sowohl in Ada und in C verfügbar sein, sodass
die Studenten die Übungen in beiden Programmiersprachen bearbeiten können.
Des Weiteren sollen die Simulationen auf der Konsole Ausgaben produzieren,
welche den aktuellen Zustand der Simulation nachvollziehen lassen. Diese
Ausgaben sollen konfigurierbar sein, sodass die Studenten auch eine eigene
Anzeige für die Anwendungen implementieren können.

% Umsetzung
\chapter{Umsetzung}
\label{chp:umsetzung}

In diesem Kapitel wird die Umsetzung der im vorigen Kapitel beschrieben
Aufgabenstellung beschrieben.

% - Druck- und Temperatursteuerung
\section{Druck- und Temperatursteuerung}
\label{sec:druck-_und_temperatursteuerung}

Für die Druck- und Temperatursteuerung wurde ein Grundgerüst in Ada und C
programmiert, welche die im vorigen Kapitel dargestellten Anforderungen erfüllt.
Dabei wurden die für die jeweilige Sprache verfügbaren Möglichkeiten zur
Umsetzung von nebenläufigen Threads und Synchronisierung verwendet, d.h.\ Tasks
und protected objects in Ada, sowie POSIX-Threads und -Mutexe in C. Es gibt
jeweils einen Thread für die Simulation von Temperatur und Druck.

Prinzipiell wurde die Simulation dabei einfach gehalten, Temperatur und Druck
ändern sich nach einem einfachen Prinzip je nach aktivierter Heizung und
Ventileinstellung. Zusätzlich werden zufällige Störungen (etwa durch äußere
Einflüsse) simuliert. Die Studenten können bereits implementierte Funktionen für
die Regelung verwenden, sodass nur die Nebenläufigkeit und der prinzipielle
Ablauf implementiert werden müssen.

Für weitere Details zur Implementierung wird hier auf die Dokumentation für
Studenten in Anhang~\ref{sec:projekt-dokumentation} verwiesen (siehe auch
Kapitel~\ref{sec:dokumentation}).

% - Parkplatz-Steuerung
\section{Parkplatz-Steuerung}
\label{sec:parkplatz-steuerung}

Das Grundgerüst für die Parkplatz-Steuerung wurde ebenfalls in Ada und C
implementiert. Dabei werden mehrere Threads erzeugt, welche die Schranken und
Autos simulieren. Dabei agieren die Autos autonom und wählen zufällig aus, ob
sie den Parkplatz betreten oder verlassen möchten (nur wenn sie bereits auf dem
Parkplatz sind). Dabei können sie Anfragen an die Threads stellen, die die
Schranken simulieren. Diese verwalten jeweils eine Warteschlage, damit die Autos
in der Reihenfolge eingelassen werden können, wie sie die Anfrage gestellt
haben. Die Umsetzung der Warteschlage erfolgte in Ada mithilfe der eingebauten
Inter-Task-Kommunikation, in C wurde mithilfe von POSIX-Mutexen,
-Bedingungsvariablen und Semaphoren eine ähnliche Funktionalität nachgebildet.

Weitere Details zur Implementierung können der Dokumentation in
Anhang~\ref{sec:projekt-dokumentation} entnommen werden (siehe auch
Kapitel~\ref{sec:dokumentation}).

% - Dokumentation
\section{Dokumentation}
\label{sec:dokumentation}

% -- Sphinx
\subsection{Sphinx}
\label{sec:sphinx}

Um die Verwendung der Simulationsumgebungen für die Studenten einfach zu
gestalten, wurde eine umfangreiche Dokumentation erstellt. Dafür wurde das
Programm \texttt{Sphinx} verwendet. Dieses erlaubt es, mithilfe von Markup-Text
Dokumentation für Software in verschiedenen Formaten zu erstellen (u.a.\ HTML
und LaTeX). Als Markup-Sprache kommt hier ReStructured Text (ReST) zum Einsatz.
Neben Text können dort Befehle verwendet werden, um Abschnitte zu untergliedern,
Bilder oder Sourcecode einzubinden und Funktionen, Prozeduren oder Datentypen zu
dokumentieren. Dabei können auch Verlinkungen und Verzeichnisse für Inhalt und
dokumentierte Funktionen/Datentypen erstellt werden.

Ein wesentlicher Vorteil der Erstellung der Dokumentation für dieses Projekt mit
Sphinx ist, dass die Dokumentation so für beide Simulationsumgebungen und in
beiden Programmiersprachen in eine einzige Webseite oder PDF-Datei (mithilfe von
LaTeX) gepackt werden konnte. Es kann somit im selben Dokument allgemein eine
Übersicht über eine Umgebung beschrieben werden und dann Code in C und Ada
dokumentiert werden, sowie beliebig auf Code in beiden Programmiersprachen
referenziert werden. Andere Programme zur Erzeugung von Dokumentation erlauben
meist nur die Verwendung von einer einzigen Programmiersprache in einem Projekt,
sodass in der Regel mehrere Webseiten oder PDF-Dokumente generiert werden.
Ebenso unterstützen die meisten gängigen Dokumentationsprogramme wie z.B.\
Doxygen Ada nicht. Mit Sphinx kann auch die Dokumentation von Funktionen oder
Datentypen innerhalb von Text erfolgen. Es ist somit sehr einfach, die
Dokumentation im Stil einer Anleitung oder eines Tutorials aufzubauen. Die
meisten Programme, welche die Dokumentation aus Kommentaren im Quellcode
aufbauen, erzeugen hingegen oft nur eine Auflistung der dokumentierten
Funktionen, ohne weiteren Text dazwischen zu erlauben. Da die Dokumentation mehr
als Anleitung zur Verwendung für die Studenten gedacht ist, ist die
Vorgehensweise von Sphinx für diese Projekt hier besser geeignet.

Leider ist die Unterstützung von Ada auch bei Sphinx nicht so gut wie für z.B.\
C oder Python. Beispielsweise wird das Index-Verzeichnis für Ada-Funktionen
nicht korrekt erstellt, außerdem werden Typ-Namen in Parameterlisten von
Prozeduren nicht automatisch verlinkt, wie das z.B.\ bei C der Fall ist. Dies
liegt vor allem daran, dass die Bibliothek \texttt{adadomain}, welche die
Unterstützung von Ada für Sphinx implementiert, zuletzt im Jahr 2013
aktualisiert wurde und nicht perfekt mehr mit der neuesten Version von Sphinx
zusammenarbeitet. Im Rahmen der Projektarbeit wurden Anpassungen an dieser
Bibliothek vorgenommen, sodass diese die automatische Verlinkung unterstützt und
die Verzeichnisse korrekt generiert werden. Dies war möglich, da Sphinx sowie
dessen Erweiterungen als Open Source vorliegen. Die Änderungen werden eventuell
noch als Pull Request in das Bitbucket-Repository eingebracht, sodass auch
andere von diesen Korrekturen und Anpassungen profitieren können. Dafür müssen
aber noch weitere Tests durchgeführt werden, da die Korrekturen nur insoweit
vorgenommen wurden, dass die Ausgabe für dieses Projekt korrekt sind. Ob dadurch
nicht weitere Inkompatibilitäten mit anderen Projekten entstanden sind, wurde
noch nicht getestet. In Kapitel~\ref{sec:kompilierung_setup} wird beschrieben,
wie die angepasste Version eingebunden werden kann. Eine Diff-Datei, welche die
vorgenommenen Änderungen enthält, ist auf der beiliegenden CD ebenfalls
enthalten (s.\ Kap.~\ref{sec:aufbau_der_verzeichnisse}).

% -- Diagramme
\subsection{Diagramme}
\label{sec:diagramme}

Innerhalb der Dokumentation (und dieses Dokuments) wurden zur
Veranschaulichung Diagramme verwendet. Diese wurden mithilfe des Programms
\texttt{plantuml} erzeugt, welches die Definition der Diagramminhalte durch
lesbaren ASCII-Text ermöglicht. Die Diagramme können damit dann als SVG (für die
HTML-Dokumentation) und als EPS oder PDF (zum Einbinden in LaTeX) generiert
werden.

% - Aufbau der Verzeichnisse
\section{Aufbau der Verzeichnisse}
\label{sec:aufbau_der_verzeichnisse}

Auf der beiliegenden CD sind unter anderem die Quelldateien für die beiden
Simulationsumgebungen sowie für die Dokumentation und diese Ausarbeitung
enthalten. Folgende Ordner sind enthalten:

\begin{description}
    \item[parking\_lot/ada] Enthält die Ada-Quellen für die Parkplatz-Simulation
    \item[parking\_lot/c] Enthält die C-Quellen für die Parkplatz-Simulation
    \item[pressure\_and\_temperature\_control/ada] Enthält die Ada-Quellen für die Druck- und Temperatursteuerung 
    \item[pressure\_and\_temperature\_control/c] Enthält die C-Quellen für die Druck- und Temperatursteuerung 
    \item[doc] Enthält die ReST-Quellen für die Dokumentation
    \item[uml] Enthält die plantuml-Quellen für die Diagramme
    \item[ausarbeitung] Enthält die LaTeX-Quellen für diese Ausarbeitung
    \item[praesentation] Enthält die LaTeX-Quellen für den Projekt-Vortrag
    \item[adadomain] Enthält die angepasste Version der
        \texttt{adadomain}-Bibliothek. Die Datei
        \texttt{changes\_janstrohbeck.diff} beinhaltet die vorgenommenen
        Änderungen.
\end{description}

% - Kompilierung, Setup
\section{Kompilierung, Setup}
\label{sec:kompilierung_setup}

Die Ada-Quellen können mit dem GNAT Compiler kompiliert werden, die C-Quellen
z.B.\ mit dem gcc-Compiler unter Verwendung des Linker-Flags \texttt{-lpthread}.

Für die Erstellung der Dokumentation ist Python 2.7 sowie Sphinx
notwendig\footnote{Eine Installationsanleitung befindet sich hier:
\url{http://www.sphinx-doc.org/en/stable/install.html}}

Die angepasste Sphinx-Erweiterung \texttt{adadomain} kann über die Kommandozeile
installiert werden. Dazu muss im \texttt{adadomain}-Verzeichnis zuerst der
folgende Befehl ausgeführt werden:

\begin{bashcode}
    python setup.py build
\end{bashcode}

Danach muss der folgende Befehl mit Administratorrechten ausgeführt werden:

\begin{bashcode}
    python setup.py install
\end{bashcode}

Die Dokumentation kann dann im Verzeichnis \texttt{doc} entweder mithilfe des
Makefiles erstellt werden (hierfür muss das Programm \texttt{make} installiert
sein) oder mithilfe der Windows-Batch-Datei \texttt{make.bat}. Dazu
müssen folgende Befehle ausgeführt werden:

\begin{bashcode}
    make html
    make latexpdf
\end{bashcode}

In den Unterordnern des Ordners \texttt{build} können dann die erzeugten
Dokumentationen (HTML und LaTeX) gefunden werden.

% Diskussion
\chapter{Diskussion}
\label{chp:diskussion}

% Fazit
\chapter{Fazit}
\label{chp:fazit}

% Bibliografie

\clearpage
%\pagestyle{plain}
%\renewcommand*{\chapterpagestyle}{plain}

%\nocite{*}
\addcontentsline{toc}{chapter}{Literatur}
\printbibliography

\appendix{}
% Anhang
\chapter{Anhang}
\label{chp:anhang}

% - Projekt-Dokumentation
\section{Projekt-Dokumentation}
\label{sec:projekt-dokumentation}

\end{document}
